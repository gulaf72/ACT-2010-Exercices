\documentclass[11pt,english,francais]{article}
\usepackage{scrtime}
\usepackage{natbib} 
\usepackage{tikz} 
\usepackage{multirow}
\usepackage[utf8]{inputenc}
\usepackage{amsmath} 
\usepackage{amsfonts} 
\usepackage{amsthm} 
\usepackage{thmtools}
\usepackage{hyperref}
\usepackage{cleveref}
\usepackage[hypcap]{caption}
\usepackage[off]{auto-pst-pdf}
\usepackage[scale=2]{ccicons}
\usepackage{tabularx}
\DeclareMathOperator{\sgn}{sgn}

\begin{document}

\section{Introduction}
\label{sec:introduction}

Ce document a pour objectif de vous introduire à l'utilisation de
Sweave, un outil qui permet de créer des rapports incorporant la
syntaxe \LaTeX et le langage statistique R

Une bonne introduction (en anglais) est faite dans ce vidéo:
\url{https://www.youtube.com/watch?v=jhj_97SNl7Y}

\end{document}
%%% Local Variables: 
%%% mode: latex
%%% TeX-master: t
%%% End: 
