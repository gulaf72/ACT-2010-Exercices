\documentclass[11pt,english,francais]{article}
\usepackage{scrtime}
\usepackage{natbib} 
\usepackage{tikz} 
\usepackage{multirow}
\usepackage[utf8]{inputenc}
\usepackage{amsmath} 
\usepackage{amsfonts} 
\usepackage{amsthm} 
\usepackage{thmtools}
\usepackage{hyperref}
\usepackage{cleveref}
\usepackage[hypcap]{caption}
\usepackage[off]{auto-pst-pdf}
\usepackage[scale=2]{ccicons}
\usepackage{tabularx}
\DeclareMathOperator{\sgn}{sgn}

\begin{document}


\section{Méthodes de lissage et saisonnalité}
\label{sec:serie-dexercices-1}

\subsection{Noël s'en vient !}
\label{sec:exercice-1-1}

On considère les taux d'inflation sur 12 mois disponibles dans le
fichier cg130823a001-fra.csv. On représente cette série chronologique
par la variable aléatoire $Y_t$. L'an passé, les Canadiens ont dépensé
en moyenne $674\$$ en cadeaux au mois de décembre 2012. Notre objectif
est de prévoir quel sera le montant dépensé pour l'achat de cadeaux en
décembre 2013.

\begin{enumerate}
\item Tracez un graphique de la série chronologique $Y_t$ à l'aide
  d'un logiciel statistique. Êtes vous en mesure de déceler
  visuellement la présence d'une tendance et/ou d'une saisonnalité ?

\item Utilisez l'opérateur différentiel $\nabla_{12}$ afin d'éliminer
  la saisonnalité annuelle de la série chronologique $Y_t$ et obtenir
  la série $Z_t$. Tracez à nouveau un graphique avec les données
  obtenues. Remarquez-vous toujours la présence de saisonnalité ?
  Tracez le graphique de la composante de saisonnalité $s_t$.

\item Maintenant, nous voulons déceler s'il y a présence d'une
  tendance dans les données. En utilisant la méthode de la moyenne
  mobile avec $q=1$ et $q=5$, du lissage exponentiel double avec
  $\alpha=5\%$ et de la régression linéaire simple, estimer la
  tendance $\hat{m}_t$. Faire le graphique superposé des 5 tendances.

\item En utilisant le résultat de la régression linéaire précédente,
  prévoir la valeur non saisonnalisée en décembre 2013. En évaluant la
  moyenne des différences entre la série $Y_t$ et la valeur de la
  régression pour les mois de décembre des années précédente, on peut
  estimer la valeur $\hat{s}_{12}$. Ajouter cette valeur au résultat
  obtenu pour obtenir une estimation du taux d'inflation en décembre
  2013.

\item En applicant ce taux d'inflation à la donnée du problème,
  prédire le montant dépensé pour l'achat de cadeaux en décembre 2013.
\end{enumerate}

\clearpage
\subsection{Incendies}

On a estimé la tendance d’un ensemble de données d’incendie pour une
année. Cependant, suite à un problème informatique, certaines données
sont manquantes. Identifiez ces données.\\

\begin{tabular}{|l|l|l|l|}
  \hline
  \multicolumn{1}{|l|}{Mois} & \multicolumn{1}{l|}{Incendies} & \multicolumn{1}{l|}{Moyenne Mobile} &  \\ \hline
  1 & 4 & \multicolumn{1}{l|}{-} &  \\ \hline
  2 & 3 & \multicolumn{1}{l|}{-} &  \\ \hline
  3 & \multicolumn{1}{l|}{a} & 4,8 &  \\ \hline
  4 & \multicolumn{1}{l|}{b} & 4,8 &  \\ \hline
  5 & 2 & 5,4 &  \\ \hline
  6 & 4 & 5,2 &  \\ \hline
  7 & 6 & 3,6 &  \\ \hline
  8 & \multicolumn{1}{l|}{c} & 3,6 &  \\ \hline
  9 & \multicolumn{1}{l|}{0} & 4,4 &  \\ \hline
  10 & 2 & 3,8 &  \\ \hline
  11 & 8 & \multicolumn{1}{l|}{-} &  \\ \hline
  12 & 3 & \multicolumn{1}{l|}{-} &  \\ \hline
\end{tabular}

\clearpage

\includegraphics[height=7mm,keepaspectratio=true]{by-sa}\\%
Cette création est mise à disposition selon le contrat
\href{http://creativecommons.org/licenses/by-sa/2.5/ca/deed.fr}{%
  Paternité-Partage à l'identique 2.5 Canada} de Creative Commons
disponible à l'adresse \\
http://creativecommons.org/licenses/by-sa/2.5/ca/deed.fr \\
En vertu de
ce contrat, vous êtes libre de:
\begin{itemize}
\item \textbf{partager} --- reproduire, distribuer et communiquer
  l'{\oe}uvre;
\item \textbf{remixer} --- adapter l'{\oe}uvre;
\item utiliser cette {\oe}uvre à des fins commerciales.
\end{itemize}
Selon les conditions suivantes:

  \begin{tabularx}{\linewidth}{@{}lX@{}}
    \raisebox{-9mm}[0mm][13mm]{%
      \includegraphics[height=11mm,keepaspectratio=true]{by}} &
    \textbf{Attribution} --- Vous devez attribuer l'{\oe}uvre de la
    manière indiquée par l'auteur de l'{\oe}uvre ou le titulaire des
    droits (mais pas d'une manière qui suggérerait qu'ils vous
    soutiennent ou
    approuvent votre utilisation de l'{\oe}uvre). \\
    \raisebox{-9mm}{\includegraphics[height=11mm,keepaspectratio=true]{sa}}
    & \textbf{Partage à l'identique} --- Si vous modifiez, transformez
    ou adaptez cette {\oe}uvre, vous n'avez le droit de distribuer
    votre création que sous une licence identique ou similaire à
    celle-ci.
  \end{tabularx}

\end{document}
%%% Local Variables: 
%%% mode: latex
%%% TeX-master: t
%%% End: 
