\documentclass{article} 
\usepackage{graphicx}
\usepackage[francais]{babel} 
\usepackage[utf8]{inputenc}
\usepackage[T1]{fontenc} 
\usepackage{amsmath} 
\usepackage{amsfonts}
\usepackage{verbatim} 
\usepackage{float} 
\usepackage{hyperref}
\usepackage{scrtime}
\usepackage[scale=2]{ccicons}
\usepackage{tabularx}
\usepackage{Sweave}
\begin{document}
\Sconcordance{concordance:serie1-solutions.tex:/home/francois/ACT-2010-Exercices/serie1-solutions.Rnw:%
1 13 1 1 0 8 1 1 2 1 0 1 2 1 0 1 2 1 1 3 0 1 2 1 1 1 2 20 0 1 8 7 1 1 2 %
19 0 1 2 1 7 7 1 1 2 19 0 1 2 1 7 11 1 1 2 19 0 1 2 1 1 1 2 19 0 1 2 1 %
1 1 2 22 0 1 3 1 1 1 2 1 0 1 1 10 0 1 1 3 0 1 3 19 0 1 2 1 10 7 1 1 2 1 %
0 2 1 7 0 1 2 2 1 1 2 1 0 1 1 3 0 1 2 39 1 1 2 1 0 4 1 1 2 1 1 1 2 1 0 %
1 1 6 0 1 3 7 0 1 1 5 0 1 4 8 0 1 2 6 0 1 2 1 0 1 1 5 0 1 2 6 0 1 1 6 0 %
1 2 114 1}


\section{Méthodes de lissage et saisonnalité}
\label{sec:serie-dexercices-1}

\subsection{Noël s'en vient !}
\label{sec:exercice-1-1}

\begin{Schunk}
\begin{Sinput}
> library(xtable) 
> library(TTR)
> Yt <- read.csv("inflation.csv",header=TRUE,sep="\t")[,2] 
> Yt.ts <-ts(Yt,start=c(2008,7),deltat=1/12) 
\end{Sinput}
\end{Schunk}

\textbf{Tableau des données} 
\begin{Schunk}
\begin{Sinput}
> xtable(Yt.ts,digits=1) 
\end{Sinput}
% latex table generated in R 2.15.2 by xtable 1.7-1 package
% Wed Sep 18 23:20:30 2013
\begin{table}[ht]
\centering
\begin{tabular}{rrrrrrrrrrrrr}
  \hline
 & Jan & Feb & Mar & Apr & May & Jun & Jul & Aug & Sep & Oct & Nov & Dec \\ 
  \hline
2008 &  &  &  &  &  &  & -0.1 & 0.5 & 0.7 & 0.9 & 1.4 & 2.3 \\ 
  2009 & 1.5 & 0.9 & 2.2 & 0.8 & 0.2 & 0.3 & 1.0 & 0.3 & 0.8 & 0.4 & 1.6 & 2.0 \\ 
  2010 & 3.2 & 2.3 & 1.4 & 0.6 & 0.7 & 1.1 & -0.2 & 1.4 & 0.9 & 1.4 & 1.4 & 1.9 \\ 
  2011 & 3.1 & 2.1 & 2.7 & 1.7 & 1.7 & 0.1 & 0.9 & 1.6 & 1.6 & 2.5 & 2.4 & 2.6 \\ 
  2012 & 2.0 & 3.2 & 2.9 & 1.4 & 1.1 & 1.3 & 1.4 & 1.4 & 1.5 & 1.7 & 2.3 & 2.4 \\ 
  2013 & 3.0 & 2.3 & 2.3 & 1.9 & 1.7 & 0.5 & 0.9 &  &  &  &  &  \\ 
   \hline
\end{tabular}
\end{table}\end{Schunk}
\begin{figure}[p]
  \centering
  \includegraphics[height=4in, width=4in]{exercice1-graph1.pdf}
  \caption{Graphique de la série $Y_t$}
  \label{fig:exercice1-graph1}
\end{figure}

\textbf{Élimination de la saisonnalité}
\begin{Schunk}
\begin{Sinput}
> xtable(Zt.ts <- diff(Yt.ts,12),digits=1)
\end{Sinput}
% latex table generated in R 2.15.2 by xtable 1.7-1 package
% Wed Sep 18 23:20:30 2013
\begin{table}[ht]
\centering
\begin{tabular}{rrrrrrrrrrrrr}
  \hline
 & Jan & Feb & Mar & Apr & May & Jun & Jul & Aug & Sep & Oct & Nov & Dec \\ 
  \hline
2009 &  &  &  &  &  &  & 1.1 & -0.2 & 0.1 & -0.5 & 0.2 & -0.3 \\ 
  2010 & 1.7 & 1.4 & -0.8 & -0.2 & 0.6 & 0.8 & -1.2 & 1.2 & 0.1 & 1.1 & -0.2 & -0.1 \\ 
  2011 & -0.1 & -0.2 & 1.4 & 1.0 & 1.0 & -0.9 & 1.1 & 0.2 & 0.7 & 1.1 & 1.0 & 0.8 \\ 
  2012 & -1.1 & 1.1 & 0.1 & -0.3 & -0.6 & 1.1 & 0.5 & -0.2 & -0.1 & -0.8 & -0.1 & -0.2 \\ 
  2013 & 1.0 & -0.9 & -0.6 & 0.5 & 0.5 & -0.8 & -0.5 &  &  &  &  &  \\ 
   \hline
\end{tabular}
\end{table}\end{Schunk}

\begin{figure}[p]
  \centering
  \includegraphics[height=4in, width=4in]{exercice1-graph2.pdf}
  \caption{Graphique de la série désaisonnalisée $Z_t$}
  \label{fig:exercice1-graph2}
\end{figure}

\textbf{Composante de saisonnalité}
\begin{Schunk}
\begin{Sinput}
> xtable(Yt.ts-Zt.ts,digits=1)
\end{Sinput}
% latex table generated in R 2.15.2 by xtable 1.7-1 package
% Wed Sep 18 23:20:30 2013
\begin{table}[ht]
\centering
\begin{tabular}{rrrrrrrrrrrrr}
  \hline
 & Jan & Feb & Mar & Apr & May & Jun & Jul & Aug & Sep & Oct & Nov & Dec \\ 
  \hline
2009 &  &  &  &  &  &  & -0.1 & 0.5 & 0.7 & 0.9 & 1.4 & 2.3 \\ 
  2010 & 1.5 & 0.9 & 2.2 & 0.8 & 0.2 & 0.3 & 1.0 & 0.3 & 0.8 & 0.4 & 1.6 & 2.0 \\ 
  2011 & 3.2 & 2.3 & 1.4 & 0.6 & 0.7 & 1.1 & -0.2 & 1.4 & 0.9 & 1.4 & 1.4 & 1.9 \\ 
  2012 & 3.1 & 2.1 & 2.7 & 1.7 & 1.7 & 0.1 & 0.9 & 1.6 & 1.6 & 2.5 & 2.4 & 2.6 \\ 
  2013 & 2.0 & 3.2 & 2.9 & 1.4 & 1.1 & 1.3 & 1.4 &  &  &  &  &  \\ 
   \hline
\end{tabular}
\end{table}\end{Schunk}

\begin{figure}[p]
  \centering
  \includegraphics[height=4in, width=4in]{exercice1-graph3.pdf}
  \caption{Graphique de la composante de saisonnalité $Y_t-Z_t$}
  \label{fig:exercice1-graph3}
\end{figure}
\clearpage
\textbf{Élimination de la tendance}

Moyenne mobile avec $q=1$
\begin{Schunk}
\begin{Sinput}
> xtable(mt1 <- lag(SMA(Zt.ts,n=3),1),digits=2)
\end{Sinput}
% latex table generated in R 2.15.2 by xtable 1.7-1 package
% Wed Sep 18 23:20:30 2013
\begin{table}[ht]
\centering
\begin{tabular}{rrrrrrrrrrrrr}
  \hline
 & Jan & Feb & Mar & Apr & May & Jun & Jul & Aug & Sep & Oct & Nov & Dec \\ 
  \hline
2009 &  &  &  &  &  &  &  & 0.32 & -0.22 & -0.06 & -0.21 & 0.54 \\ 
  2010 & 0.92 & 0.77 & 0.15 & -0.12 & 0.39 & 0.07 & 0.26 & 0.05 & 0.78 & 0.32 & 0.22 & -0.16 \\ 
  2011 & -0.13 & 0.36 & 0.74 & 1.12 & 0.36 & 0.37 & 0.10 & 0.63 & 0.63 & 0.90 & 0.93 & 0.20 \\ 
  2012 & 0.23 & 0.03 & 0.32 & -0.24 & 0.10 & 0.37 & 0.51 & 0.09 & -0.37 & -0.34 & -0.37 & 0.25 \\ 
  2013 & -0.02 & -0.16 & -0.33 & 0.16 & 0.10 & -0.26 &  &  &  &  &  &  \\ 
   \hline
\end{tabular}
\end{table}\end{Schunk}
\clearpage 
Moyenne mobile avec $q=5$
\begin{Schunk}
\begin{Sinput}
> xtable(mt2 <- lag(SMA(Zt.ts,n=11),5),digits=2)
\end{Sinput}
% latex table generated in R 2.15.2 by xtable 1.7-1 package
% Wed Sep 18 23:20:30 2013
\begin{table}[ht]
\centering
\begin{tabular}{rrrrrrrrrrrrr}
  \hline
 & Jan & Feb & Mar & Apr & May & Jun & Jul & Aug & Sep & Oct & Nov & Dec \\ 
  \hline
2009 &  &  &  &  &  &  &  &  &  &  &  & 0.28 \\ 
  2010 & 0.25 & 0.17 & 0.26 & 0.32 & 0.40 & 0.40 & 0.24 & 0.10 & 0.16 & 0.30 & 0.34 & 0.36 \\ 
  2011 & 0.37 & 0.37 & 0.37 & 0.33 & 0.45 & 0.55 & 0.63 & 0.54 & 0.52 & 0.44 & 0.32 & 0.36 \\ 
  2012 & 0.36 & 0.40 & 0.32 & 0.22 & 0.05 & -0.02 & 0.06 & 0.06 & -0.04 & -0.07 & 0.03 & -0.02 \\ 
  2013 & -0.14 & -0.18 &  &  &  &  &  &  &  &  &  &  \\ 
   \hline
\end{tabular}
\end{table}\end{Schunk}
\clearpage 
Lissage exponentiel double avec $\alpha=0.75$
\begin{Schunk}
\begin{Sinput}
> xtable(mt3 <- DEMA(Zt.ts,n=1,ratio=.05),digits=2)
\end{Sinput}
% latex table generated in R 2.15.2 by xtable 1.7-1 package
% Wed Sep 18 23:20:30 2013
\begin{table}[ht]
\centering
\begin{tabular}{rrrrrrrrrrrrr}
  \hline
 & Jan & Feb & Mar & Apr & May & Jun & Jul & Aug & Sep & Oct & Nov & Dec \\ 
  \hline
2009 &  &  &  &  &  &  & 1.06 & 0.94 & 0.85 & 0.71 & 0.66 & 0.55 \\ 
  2010 & 0.65 & 0.72 & 0.57 & 0.48 & 0.48 & 0.50 & 0.33 & 0.39 & 0.36 & 0.41 & 0.33 & 0.28 \\ 
  2011 & 0.22 & 0.17 & 0.27 & 0.33 & 0.38 & 0.24 & 0.31 & 0.29 & 0.31 & 0.38 & 0.43 & 0.45 \\ 
  2012 & 0.29 & 0.36 & 0.33 & 0.26 & 0.17 & 0.25 & 0.27 & 0.22 & 0.18 & 0.07 & 0.04 & 0.01 \\ 
  2013 & 0.09 & -0.02 & -0.09 & -0.04 & 0.00 & -0.09 & -0.14 &  &  &  &  &  \\ 
   \hline
\end{tabular}
\end{table}\end{Schunk}
\clearpage 
Régression linéaire
\begin{Schunk}
\begin{Sinput}
> t <- 0:48
> (lm1 <- lm(Zt.ts~t))
\end{Sinput}
\begin{Soutput}
Call:
lm(formula = Zt.ts ~ t)

Coefficients:
(Intercept)            t  
   0.446924    -0.009856  
\end{Soutput}
\begin{Sinput}
> coeff1 <- coefficients(lm1)
\end{Sinput}
\end{Schunk}
\begin{Schunk}
\begin{Sinput}
> xtable(mt4 <- ts(coeff1[1]+t*coeff1[2],start=c(2009,7),deltat=1/12),digits=2)
\end{Sinput}
% latex table generated in R 2.15.2 by xtable 1.7-1 package
% Wed Sep 18 23:20:30 2013
\begin{table}[ht]
\centering
\begin{tabular}{rrrrrrrrrrrrr}
  \hline
 & Jan & Feb & Mar & Apr & May & Jun & Jul & Aug & Sep & Oct & Nov & Dec \\ 
  \hline
2009 &  &  &  &  &  &  & 0.45 & 0.44 & 0.43 & 0.42 & 0.41 & 0.40 \\ 
  2010 & 0.39 & 0.38 & 0.37 & 0.36 & 0.35 & 0.34 & 0.33 & 0.32 & 0.31 & 0.30 & 0.29 & 0.28 \\ 
  2011 & 0.27 & 0.26 & 0.25 & 0.24 & 0.23 & 0.22 & 0.21 & 0.20 & 0.19 & 0.18 & 0.17 & 0.16 \\ 
  2012 & 0.15 & 0.14 & 0.13 & 0.12 & 0.11 & 0.10 & 0.09 & 0.08 & 0.07 & 0.06 & 0.05 & 0.04 \\ 
  2013 & 0.03 & 0.02 & 0.01 & 0.00 & -0.01 & -0.02 & -0.03 &  &  &  &  &  \\ 
   \hline
\end{tabular}
\end{table}\end{Schunk}
\clearpage 
\begin{figure}[p]
  \centering
  \includegraphics[height=4in, width=4in]{exercice1-graph4.pdf}
  \caption{Graphique de la tendance $m_t$}
  \label{fig:exercice1-graph4}
\end{figure}
\clearpage 
\textbf{Projection du taux d'inflation}
\begin{Schunk}
\begin{Sinput}
> projection <- coeff1[1]+53*coeff1[2]
> saisonnalite <- mean((Yt.ts-Zt.ts)[6+12*0:3])
> (taux.inf.dec.2013 <- (projection+saisonnalite))
\end{Sinput}
\begin{Soutput}
(Intercept) 
   2.137743 
\end{Soutput}
\end{Schunk}
Le taux d'inflation prejeté en décembre 2013 est 2.14\%\\

\textbf{Solution du problème}
\begin{Schunk}
\begin{Sinput}
> depense.dec.2008 <- 674
> depense.dec.2013 <- 674*(1+taux.inf.dec.2013/100)
\end{Sinput}
\end{Schunk}
Le montant projeté des achats de cadeaux en décembre 2013 est 688.41 \$

\clearpage 
\subsection{Incendies}

On remarque d'abord que $q=2$.

On peut ensuite poser les équations suivantes:
\begin{align}
  \label{eq:1}
  4+3+a+b+2 &= 24\\
  b+2+4+6+c &= 26\\
  c+0+2+8+3 &= 19
\end{align}

En résolvant, on obtient la solution.\\

\textbf{Solution:}\\

\begin{tabular}{|l|l|l|}
\hline
\multicolumn{1}{|l|}{Mois} & \multicolumn{1}{l|}{Incendies} & \multicolumn{1}{l|}{Moyenne Mobile} \\ \hline
1 & 4 & \multicolumn{1}{l|}{-} \\ \hline
2 & 3 & \multicolumn{1}{l|}{-} \\ \hline
3 & \textbf{7} & 4,8 \\ \hline
4 & \textbf{8} & 4,8 \\ \hline
5 & 2 & 5,4 \\ \hline
6 & 4 & 5,2 \\ \hline
7 & 6 & 3,6 \\ \hline
8 & \textbf{6} & 3,6 \\ \hline
9 & 0  & 4,4 \\ \hline
10 & 2 & 3,8 \\ \hline
11 & 8 & \multicolumn{1}{l|}{-} \\ \hline
12 & 3 & \multicolumn{1}{l|}{-} \\ \hline
\end{tabular}

\clearpage

Document généré le \today \ à \ \thistime 

\clearpage

\includegraphics[height=7mm,keepaspectratio=true]{by-sa}\\
Cette création est mise à disposition selon le contrat
\href{http://creativecommons.org/licenses/by-sa/2.5/ca/deed.fr}{%
  Paternité-Partage à l'identique 2.5 Canada} de Creative Commons
disponible à l'adresse \\
http://creativecommons.org/licenses/by-sa/2.5/ca/deed.fr \\
En vertu de ce contrat, vous êtes libre de:
\begin{itemize}
\item \textbf{partager} --- reproduire, distribuer et communiquer
  l'{\oe}uvre;
\item \textbf{remixer} --- adapter l'{\oe}uvre;
\item utiliser cette {\oe}uvre à des fins commerciales.
\end{itemize}
Selon les conditions suivantes:

  \begin{tabularx}{\linewidth}{@{}lX@{}}
    \raisebox{-9mm}[0mm][13mm]{%
      \includegraphics[height=11mm,keepaspectratio=true]{by}} &
    \textbf{Attribution} --- Vous devez attribuer l'{\oe}uvre de la
    manière indiquée par l'auteur de l'{\oe}uvre ou le titulaire des
    droits (mais pas d'une manière qui suggérerait qu'ils vous
    soutiennent ou
    approuvent votre utilisation de l'{\oe}uvre). \\
    \raisebox{-9mm}{\includegraphics[height=11mm,keepaspectratio=true]{sa}}
    & \textbf{Partage à l'identique} --- Si vous modifiez, transformez
    ou adaptez cette {\oe}uvre, vous n'avez le droit de distribuer
    votre création que sous une licence identique ou similaire à
    celle-ci.
  \end{tabularx}
\end{document}
