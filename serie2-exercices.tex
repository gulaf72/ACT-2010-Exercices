\documentclass[11pt,english,francais]{article}
\usepackage{scrtime}
\usepackage{natbib} 
\usepackage{tikz} 
\usepackage{multirow}
\usepackage[utf8]{inputenc}
\usepackage{amsmath} 
\usepackage{amsfonts} 
\usepackage{amsthm} 
\usepackage{thmtools}
\usepackage{hyperref}
\usepackage{cleveref}
\usepackage[hypcap]{caption}
\usepackage[off]{auto-pst-pdf}
\usepackage[scale=2]{ccicons}
\usepackage{tabularx}
\DeclareMathOperator{\sgn}{sgn}

\begin{document}

\begin{center}
\thispagestyle{empty}
{\LARGE ACT-2010 \\
\vspace*{\baselineskip}
Séries Chronologiques}
\rule{\textwidth}{0.8pt}\\[\baselineskip]
\vspace*{16\baselineskip}
{\LARGE Exercices et solutions} \\
\vfill
\rule{\textwidth}{0.8pt}\\[\baselineskip]
{\large Mis à jour le \today \\
François Pelletier \\
École d'Actuariat, Université Laval}
\end{center}
\clearpage
\setcounter{page}{1}
%%% Local Variables: 
%%% mode: latex
%%% TeX-master: t
%%% End: 


\addtocounter{section}{1}

\section{Modèles classiques pour séries chronologiques}

\subsection{Zone de stationnarité pour AR(2) (Théorique)}
\label{sec:zone-de-stat}


Pour avoir la stationnarité, il faut que les racines du polynôme caractéristique soient inférieures à 1 en valeur absolue. Démontrez que pour le modèle AR(2), la stationnarité est possible si et seulement si les trois conditions suivantes sont réunies:
\begin{align*}
  \phi_1 + \phi_2 &< 1 \\
  \phi_2 - \phi_1 &< 1 \\
  |\phi_2| &< 1
\end{align*}

\subsection{Ordre d'intégration (Théorique)}
\label{sec:ordre-dintegration}

\begin{enumerate}
\item 
Un polynôme d'ordre $k$ en t est intégré d'ordre $k$ puisque
\begin{align*}
  (1-B)^k(a_0+a_1t+a_2t^2+\ldots+a_kt^k) &= k!a_k 
\end{align*}

Démontrez cette affirmation.
\item 

Démontrez que si $x_t$ est stationnaire, alors $(1-B)x_t$ est aussi stationnaire.
\end{enumerate}

\subsection{Inversion de processus d'ordre 1 (Théorique)}
\label{sec:invers-de-proc}

\begin{enumerate}
\item 
Démontrez algébriquement qu'un processus AR(1) est équivalent à un processus MA($\infty$).

\item 
Démontrez algébriquement qu'un processus MA(1) est équivalent à un processus AR($\infty$).
\end{enumerate}

\subsection{Construction d'une série autorégressive (Calculatrice)}
\label{sec:constr-dune-serie}

On considère les 10 nombres aléatoires suivants, issus d'une distribution normale centrée réduite:

\begin{verbatim}
 [1] -1.21  0.28  1.08 -2.35  0.43  0.51 -0.57 -0.55 -0.56 -0.89
\end{verbatim}

Construisez la série autorégressive d'ordre 1 avec coefficient: 
\begin{enumerate}
\item $\phi = -0.5$
\item $\phi = 0.5$
\end{enumerate}

Quelle différence observez-vous entre la série avec une corrélation négative et la série avec une corrélation positive ?

\subsection{Deux processus MA(2) (Théorique)}

On considère deux processus MA(2), un où $\theta_1 = \theta_2 = \frac{1}{4}$, et un autre où $\theta_1=-1$ et $\theta_2 = 4$. Démontrez que ces processus ont la même fonction d'autocorrélation.

\subsection{Estimateur des moments pour le processus AR(2) (Théorique et calculatrice)}

\begin{enumerate}
\item En utilisant les équations de Yule-Walker, dérivez un estimateur des moments pour les paramètres $\phi_1$ et $\phi_1$ d'un processus AR(2). \\

\item Estimez les paramètres du processus AR(2) à partir de la série suivante:
\begin{verbatim}
 [1]  1.1617660  0.6981185  0.1693004 -0.6457205  1.4217278  1.3701445
 [7] -1.6369769 -0.4596686 -0.2933815 -1.0995973
\end{verbatim}
\end{enumerate}

\subsection{Terme d'erreur du processus ARMA(2,1) (Théorique)}

\begin{enumerate}
\item Démontrez que le terme d'erreur $\epsilon_t$ d'un processus ARMA(2,1) peut être exprimé sous la forme suivante, où $\mu$ est une constante et $\phi_1, \phi_2, \theta$ sont les paramètres du modèle. On considère que la série est stationnaire.
\begin{align*}
  \epsilon_t = \sum_{i=0}^{\infty} \theta^i \left(y_{t-i} - \mu - \phi_1 y_{t-i-1} - \phi_2 y_{t-i-2} \right)
\end{align*}

\item De plus, démontrez qu'à partir de cette forme du terme d'erreur, on peut obtenir la représentation AR($\infty$) du processus ARMA(2,1).
\end{enumerate}

\subsection{Racines caractéristiques et prévision}

On considère l'équation en différence suivante:
\begin{align*}
  y_t = 1.5 y_{t-1} - 0.5 y_{t-2} + \epsilon_t
\end{align*}
\begin{enumerate}
\item À quel modèle correspond cette équation ?
\item Trouvez les racines de l'équation homogène.
\item Démontrez que les racines de l'équation $1-1.5B+0.5B^2$ sont la réciproque des valeurs trouvées à la question précédente.
\item Est-ce que cette série est stationnaire ?
\item On suppose que l'on connaît les deux premiers termes de la série $y_0$ et $y_1$. Trouvez la solution générale pour $y_t$ en fonction de la séquence des valeurs de $\epsilon_t$.
\item Identifiez la forme de la fonction de prédiction pour $y_{T+s}$, sachant les valeurs de $y_{T-1}$ et $y_T$.
\item Évaluez $E[y_t]$, $E[y_{t+1}]$, $Var[y_t]$, $Var[y_{t+1}]$ et $Cov[y_{t},y_{t+1}]$.
\item Donnez l'expression d'un intervalle de confiance à 95\% pour la valeur de $y_{t+1}$ 
\end{enumerate}

\clearpage

\input{cc}
\end{document}
%%% Local Variables: 
%%% mode: latex
%%% TeX-master: t
%%% End: 
