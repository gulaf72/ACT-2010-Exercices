\documentclass[11pt,english,francais]{article}
\usepackage{scrtime}
\usepackage{natbib} 
\usepackage{tikz} 
\usepackage{multirow}
\usepackage[utf8]{inputenc}
\usepackage{amsmath} 
\usepackage{amsfonts} 
\usepackage{amsthm} 
\usepackage{thmtools}
\usepackage{hyperref}
\usepackage{cleveref}
\usepackage[hypcap]{caption}
\usepackage[off]{auto-pst-pdf}
\usepackage[scale=2]{ccicons}
\usepackage{tabularx}
\DeclareMathOperator{\sgn}{sgn}

\begin{document}
\addtocounter{section}{1}

\section{Modèles classiques pour séries chronologiques}

\subsection{Zone de stationnarité pour AR(2) (Théorique)}
\label{sec:zone-de-stat}


Pour avoir la stationnarité, il faut que les racines du polynôme caractéristique soient inférieures à 1 en valeur absolue. Démontrez que pour le modèle AR(2), la stationnarité est possible si et seulement si les trois conditions suivantes sont réunies:
\begin{align*}
  \phi_1 + \phi_2 &< 1 \\
  \phi_2 - \phi_1 &< 1 \\
  |\phi_2| &< 1
\end{align*}

\subsection{Ordre d'intégration (Théorique)}
\label{sec:ordre-dintegration}

\begin{enumerate}
\item 
Un polynôme d'ordre $k$ en t est intégré d'ordre $k$ puisque
\begin{align*}
  (1-B)^k(a_0+a_1t+a_2t^2+\ldots+a_kt^k) &= k!a_k 
\end{align*}

Démontrez cette affirmation.
\item 

Démontrez que si $x_t$ est stationnaire, alors $(1-B)x_t$ est aussi stationnaire.
\end{enumerate}

\subsection{Inversion de processus d'ordre 1 (Théorique)}
\label{sec:invers-de-proc}

\begin{enumerate}
\item 
Démontrez algébriquement qu'un processus AR(1) est équivalent à un processus MA($\infty$).

\item 
Démontrez algébriquement qu'un processus MA(1) est équivalent à un processus AR($\infty$).
\end{enumerate}

\subsection{Construction d'une série autorégressive (Calculatrice)}
\label{sec:constr-dune-serie}

On considère les 10 nombres aléatoires suivants, issus d'une distribution normale centrée réduite:

\begin{verbatim}
 [1] -1.21  0.28  1.08 -2.35  0.43  0.51 -0.57 -0.55 -0.56 -0.89
\end{verbatim}

Construisez la série autorégressive d'ordre 1 avec coefficient: 
\begin{enumerate}
\item $\gamma = -0.5$
\item $\gamma = 0.5$
\end{enumerate}

Quelle différence observez-vous entre la série avec une corrélation négative et la série avec une corrélation positive ?

\clearpage

\input{cc}


\end{document}
%%% Local Variables: 
%%% mode: latex
%%% TeX-master: t
%%% End: 
