\chapter{Modèles de volatilité stochastique}
\label{chap:modeles-volatilite}

\Opensolutionfile{reponses}[reponses-modeles-volatilite]
\Opensolutionfile{solutions}[solutions-modeles-volatilite]

\begin{Filesave}{reponses}
\bigskip
\section*{Réponses}

\end{Filesave}

\begin{Filesave}{solutions}
\section*{Chapitre \ref{chap:modeles-volatilite}}
\addcontentsline{toc}{section}{Chapitre \protect\ref{chap:modeles-volatilite}}

\end{Filesave}


\begin{exercice}
  On considère un processus ARCH(2) dont le carré des résidus répond à l'équation suivante \footnote{Cet exercice est inspiré de l'exercice 8 du chapitre 3 de Enders (2004)} :
\begin{align*}
\epsilon_t^2 &= \alpha_0 + \alpha_1\epsilon_{t-1}^2 + \alpha_2\epsilon_{t-2}^2.
\end{align*}

On suppose que les résidus proviennent du modèle suivant:
\begin{align*}
y_t &= a_0 + a_1 y_{t-1} + \epsilon_t.
\end{align*}

Trouvez la variance conditionnelle et inconditionnelle de $\left\{ y_t \right\}$.
\begin{sol}
  On identifie d'abord la moyenne conditionnelle de $y_t$:
\begin{align}
E_{t-1}[y_t] &= E_{t-1}[a_0 + a_1 y_{t-1} + \epsilon_t] \\
&= a_0 + a_1 y_{t-1}
\end{align}
La variance conditionnelle peut alors s'obtenir en utilisant la définition habituelle:
\begin{align*}
V_{t-1}[y_t | y_{t-1}, y_{t-2}, \ldots] &= E_{t-1}[y_t - E_{t-1}[y_t]]^{2} \\
&= E_{t-1}[(a_0 + a_1 y_{t-1} + \epsilon_t)-(a_0 + a_1 y_{t-1})]^{2} \\
&= E_{t-1}[\epsilon_t]^{2} \\
&= E_{t-1}[\alpha_0 + \alpha_1\epsilon_{t-1}^2 + \alpha_2\epsilon_{t-2}^2] \\
&= \alpha_0 + \alpha_1\epsilon_{t-1}^2 + \alpha_2\epsilon_{t-2}^2
\end{align*}

La variance inconditionnelle s'obtient en trouvant la solution particulière pour $y_t$:
\begin{align*}
•y_t &= a_0 + a_1 y_{t-1} + \epsilon_t \\
&= (1+a_1)a_0 + a_1^2 y_{t-2} + a_1 \epsilon_{t-1} + \epsilon_t \\
&= \cdots \\
&= (a+a_1+a_2+a_3+\ldots)a_0 + \epsilon_t + a_1\epsilon_{t-1} + a_2\epsilon_{t-2} + \ldots \\
&= \frac{a_0}{1-a_1} + \sum_{i=0}^{\infty} a_1^{i}\epsilon_{t-i}
\end{align*}

On évalue la variance de cette dernière expression:
\begin{align*}
•Var[y_t] &= Var[\sum_{i=0}^{\infty} a_1^{i}\epsilon_{t-i}] \\
&= \sum_{i=0}^{\infty} a_1^{2i} Var[\epsilon_{t-i}] \\
&= \frac{\sigma^2}{1-a_1^2}
\end{align*}

À partir de la définition, on a que:
\begin{align*}
E[\epsilon_t^2] &= \alpha_0 + \alpha_1 E_[\epsilon_{t-1}^2] + \alpha_2 E[\epsilon_{t-2}^2].
\end{align*}

Comme la variance inconditionnelle de $\epsilon_t$ est identique à celle de $\epsilon_{t-1}$ et $\epsilon_{t-2}$, on peut affirmer que:
\begin{align*}
E[\epsilon_t^2] &= \frac{\alpha_0}{1-\alpha_1-\alpha_2} \\
&= \sigma^2.
\end{align*}

On obtient donc que la variance inconditionnelle de $y_t$ est
\begin{align*}
•Var[y_t] &= \frac{\alpha_0}{(1-\alpha_1-\alpha_2)(1-a_1^2)}.
\end{align*}
\end{sol}
\end{exercice}

\Closesolutionfile{solutions}
\Closesolutionfile{reponses}

%%% Local Variables: 
%%% mode: latex
%%% TeX-master: "exercices_series_chrono"
%%% End:
