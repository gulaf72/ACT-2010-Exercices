\documentclass[11pt,english,francais]{article}
\usepackage{scrtime}
\usepackage{natbib} 
\usepackage{tikz} 
\usepackage{multirow}
\usepackage[utf8]{inputenc}
\usepackage{amsmath} 
\usepackage{amsfonts} 
\usepackage{amsthm} 
\usepackage{thmtools}
\usepackage{hyperref}
\usepackage{cleveref}
\usepackage[hypcap]{caption}
\usepackage[off]{auto-pst-pdf}
\usepackage[scale=2]{ccicons}
\usepackage{tabularx}
\DeclareMathOperator{\sgn}{sgn}

\begin{document}

\begin{center}
\thispagestyle{empty}
{\LARGE ACT-2010 \\
\vspace*{\baselineskip}
Séries Chronologiques}
\rule{\textwidth}{0.8pt}\\[\baselineskip]
\vspace*{16\baselineskip}
{\LARGE Exercices et solutions} \\
\vfill
\rule{\textwidth}{0.8pt}\\[\baselineskip]
{\large Mis à jour le \today \\
François Pelletier \\
École d'Actuariat, Université Laval}
\end{center}
\clearpage
\setcounter{page}{1}
%%% Local Variables: 
%%% mode: latex
%%% TeX-master: t
%%% End: 


\addtocounter{section}{2}

\section{Modèles de volatilité stochastique}
\label{sec:serie-dexercices-3}

\subsection{Variance du processus ARCH(2)}
\label{sec:exercice-3-1}

On considère un processus ARCH(2) dont le carré des résidus répond à l'équation suivante \footnote{Cet exercice est inspiré de l'exercice 8 du chapitre 3 de Enders (2004)} :
\begin{align*}
\epsilon_t^2 &= \alpha_0 + \alpha_1\epsilon_{t-1}^2 + \alpha_2\epsilon_{t-2}^2.
\end{align*}

On suppose que les résidus proviennent du modèle suivant:
\begin{align*}
y_t &= a_0 + a_1 y_{t-1} + \epsilon_t.
\end{align*}

Trouvez la variance conditionnelle et inconditionnelle de $\left\{ y_t \right\}$.

\clearpage

\input{cc}
\end{document}
%%% Local Variables: 
%%% mode: latex
%%% TeX-master: t
%%% End: 
